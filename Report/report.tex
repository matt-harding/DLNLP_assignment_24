\documentclass{article}
\usepackage{natbib}
\usepackage{url}
\usepackage{amsmath}
\usepackage{tabularx}
\usepackage{hyperref}
\usepackage{graphicx}
\graphicspath{ {./images/} } 

\title{Thematic classification of questions}
\author{Matthew Harding [Student ID - MHARD94]}
\date{May 2024}

\begin{document}

\maketitle

\section{Abstract}


\section{Introduction}

The aim of this project was to benchmark the performance of several techniques for classifying the theme of a question. The hypothesis being that 
the wording of a questions, without the addition of other context or metadata, is sufficient to classify a question to a general theme.
 

\section{Training Data}

The Yahoo! Answers topic classification dataset \cite{huggingface_dataset} is a set of set of questions grouped by topic taken from the Yahoo! Answers corpus as of 10/25/2007.
The data was obtained through the Yahoo! Research Alliance Webscope program \cite{yahoo_webscope}.


The dataset contains a training set 1,400,000 questions. Luckily, the dataset is evenly split across the 10 categories so there is no issue of class imbalance. 
60,000 additional questions are reserved for testing. Again the test questions are evenly split across the categories. 

\begin{center}
    \begin{tabular}{||c c c c||} 
     \hline
     Topic Code & Topic Label & Training Count & Test Count \\ [0.5ex] 
     \hline\hline
     0 & Society \& Culture & 140000 & 6000 \\ 
     \hline
     1 & Science \& Mathematics & 140000 & 6000 \\ 
     \hline
     2 & Health & 140000 & 6000 \\
     \hline
     3 & Education \& Reference & 140000 & 6000 \\
     \hline
     4 & Computers \& Internet & 140000 & 6000 \\
     \hline
     5 & Sports & 140000 & 6000 \\ 
     \hline
     6 & Business \& Finance & 140000 & 6000 \\ 
     \hline
     7 & Entertainment \& Music & 140000 & 6000 \\
     \hline
     8 & Family \& Relationships & 140000 & 6000 \\
     \hline
     9 & Polotics \& Government & 140000 & 6000 \\ [1ex] 
     \hline
    \end{tabular}
    \end{center}

The **Datasets** library from Huggingface was used to pull the data. For each question, there is a a topic label, question title, question content and best answer. For 
this project, we consider the topic to combination of question title and question content to be the classification input and the topic label to be the classifcation output. 
The best answer value was disregarded for this task.

\bibliographystyle{plain}
\bibliography{references}

\end{document}
